
%% bare_conf.tex
%% V1.4b
%% 2015/08/26
%% by Michael Shell
%% See:
%% http://www.michaelshell.org/
%% for current contact information.
%%
%% This is a skeleton file demonstrating the use of IEEEtran.cls
%% (requires IEEEtran.cls version 1.8b or later) with an IEEE
%% conference paper.
%%
%% Support sites:
%% http://www.michaelshell.org/tex/ieeetran/
%% http://www.ctan.org/pkg/ieeetran
%% and
%% http://www.ieee.org/

%%*************************************************************************
%% Legal Notice:
%% This code is offered as-is without any warranty either expressed or
%% implied; without even the implied warranty of MERCHANTABILITY or
%% FITNESS FOR A PARTICULAR PURPOSE! 
%% User assumes all risk.
%% In no event shall the IEEE or any contributor to this code be liable for
%% any damages or losses, including, but not limited to, incidental,
%% consequential, or any other damages, resulting from the use or misuse
%% of any information contained here.
%%
%% All comments are the opinions of their respective authors and are not
%% necessarily endorsed by the IEEE.
%%
%% This work is distributed under the LaTeX Project Public License (LPPL)
%% ( http://www.latex-project.org/ ) version 1.3, and may be freely used,
%% distributed and modified. A copy of the LPPL, version 1.3, is included
%% in the base LaTeX documentation of all distributions of LaTeX released
%% 2003/12/01 or later.
%% Retain all contribution notices and credits.
%% ** Modified files should be clearly indicated as such, including  **
%% ** renaming them and changing author support contact information. **
%%*************************************************************************


% *** Authors should verify (and, if needed, correct) their LaTeX system  ***
% *** with the testflow diagnostic prior to trusting their LaTeX platform ***
% *** with production work. The IEEE's font choices and paper sizes can   ***
% *** trigger bugs that do not appear when using other class files.       ***                          ***
% The testflow support page is at:
% http://www.michaelshell.org/tex/testflow/


\documentclass[conference]{IEEEtran}
% Some Computer Society conferences also require the compsoc mode option,
% but others use the standard conference format.
%
% If IEEEtran.cls has not been installed into the LaTeX system files,
% manually specify the path to it like:
% \documentclass[conference]{../sty/IEEEtran}


\usepackage[utf8]{inputenc}	% arquivos LaTeX em Unicode (UTF8)
\RequirePackage[plainpages,pdfpagelabels]{hyperref}	% PDF com links, metadados


% Some very useful LaTeX packages include:
% (uncomment the ones you want to load)


% *** MISC UTILITY PACKAGES ***
%
%\usepackage{ifpdf}
% Heiko Oberdiek's ifpdf.sty is very useful if you need conditional
% compilation based on whether the output is pdf or dvi.
% usage:
% \ifpdf
%   % pdf code
% \else
%   % dvi code
% \fi
% The latest version of ifpdf.sty can be obtained from:
% http://www.ctan.org/pkg/ifpdf
% Also, note that IEEEtran.cls V1.7 and later provides a builtin
% \ifCLASSINFOpdf conditional that works the same way.
% When switching from latex to pdflatex and vice-versa, the compiler may
% have to be run twice to clear warning/error messages.






% *** CITATION PACKAGES ***
%
%\usepackage{cite}
% cite.sty was written by Donald Arseneau
% V1.6 and later of IEEEtran pre-defines the format of the cite.sty package
% \cite{} output to follow that of the IEEE. Loading the cite package will
% result in citation numbers being automatically sorted and properly
% "compressed/ranged". e.g., [1], [9], [2], [7], [5], [6] without using
% cite.sty will become [1], [2], [5]--[7], [9] using cite.sty. cite.sty's
% \cite will automatically add leading space, if needed. Use cite.sty's
% noadjust option (cite.sty V3.8 and later) if you want to turn this off
% such as if a citation ever needs to be enclosed in parenthesis.
% cite.sty is already installed on most LaTeX systems. Be sure and use
% version 5.0 (2009-03-20) and later if using hyperref.sty.
% The latest version can be obtained at:
% http://www.ctan.org/pkg/cite
% The documentation is contained in the cite.sty file itself.






% *** GRAPHICS RELATED PACKAGES ***
%
\ifCLASSINFOpdf
  \usepackage[pdftex]{graphicx}
  % declare the path(s) where your graphic files are
  \graphicspath{{../pdf/}{../jpeg/}{./plots/}}
  % and their extensions so you won't have to specify these with
  % every instance of \includegraphics
  \DeclareGraphicsExtensions{.pdf,.jpeg,.png}
\else
  % or other class option (dvipsone, dvipdf, if not using dvips). graphicx
  % will default to the driver specified in the system graphics.cfg if no
  % driver is specified.
  % \usepackage[dvips]{graphicx}
  % declare the path(s) where your graphic files are
  % \graphicspath{{../eps/}}
  % and their extensions so you won't have to specify these with
  % every instance of \includegraphics
  % \DeclareGraphicsExtensions{.eps}
\fi
% graphicx was written by David Carlisle and Sebastian Rahtz. It is
% required if you want graphics, photos, etc. graphicx.sty is already
% installed on most LaTeX systems. The latest version and documentation
% can be obtained at: 
% http://www.ctan.org/pkg/graphicx
% Another good source of documentation is "Using Imported Graphics in
% LaTeX2e" by Keith Reckdahl which can be found at:
% http://www.ctan.org/pkg/epslatex
%
% latex, and pdflatex in dvi mode, support graphics in encapsulated
% postscript (.eps) format. pdflatex in pdf mode supports graphics
% in .pdf, .jpeg, .png and .mps (metapost) formats. Users should ensure
% that all non-photo figures use a vector format (.eps, .pdf, .mps) and
% not a bitmapped formats (.jpeg, .png). The IEEE frowns on bitmapped formats
% which can result in "jaggedy"/blurry rendering of lines and letters as
% well as large increases in file sizes.
%
% You can find documentation about the pdfTeX application at:
% http://www.tug.org/applications/pdftex





% *** MATH PACKAGES ***
%
%\usepackage{amsmath}
% A popular package from the American Mathematical Society that provides
% many useful and powerful commands for dealing with mathematics.
%
% Note that the amsmath package sets \interdisplaylinepenalty to 10000
% thus preventing page breaks from occurring within multiline equations. Use:
%\interdisplaylinepenalty=2500
% after loading amsmath to restore such page breaks as IEEEtran.cls normally
% does. amsmath.sty is already installed on most LaTeX systems. The latest
% version and documentation can be obtained at:
% http://www.ctan.org/pkg/amsmath



\usepackage{cleveref}

% *** SPECIALIZED LIST PACKAGES ***
%
\usepackage{algorithm}
\usepackage[noend]{algpseudocode}
\floatname{algorithm}{Algorithm}
\renewcommand{\algorithmiccomment}[1]{~~~// #1}
%\usepackage{algorithmic}
% algorithmic.sty was written by Peter Williams and Rogerio Brito.
% This package provides an algorithmic environment fo describing algorithms.
% You can use the algorithmic environment in-text or within a figure
% environment to provide for a floating algorithm. Do NOT use the algorithm
% floating environment provided by algorithm.sty (by the same authors) or
% algorithm2e.sty (by Christophe Fiorio) as the IEEE does not use dedicated
% algorithm float types and packages that provide these will not provide
% correct IEEE style captions. The latest version and documentation of
% algorithmic.sty can be obtained at:
% http://www.ctan.org/pkg/algorithms
% Also of interest may be the (relatively newer and more customizable)
% algorithmicx.sty package by Szasz Janos:
% http://www.ctan.org/pkg/algorithmicx




% *** ALIGNMENT PACKAGES ***
%
%\usepackage{array}
% Frank Mittelbach's and David Carlisle's array.sty patches and improves
% the standard LaTeX2e array and tabular environments to provide better
% appearance and additional user controls. As the default LaTeX2e table
% generation code is lacking to the point of almost being broken with
% respect to the quality of the end results, all users are strongly
% advised to use an enhanced (at the very least that provided by array.sty)
% set of table tools. array.sty is already installed on most systems. The
% latest version and documentation can be obtained at:
% http://www.ctan.org/pkg/array


% IEEEtran contains the IEEEeqnarray family of commands that can be used to
% generate multiline equations as well as matrices, tables, etc., of high
% quality.




% *** SUBFIGURE PACKAGES ***
%\ifCLASSOPTIONcompsoc
%  \usepackage[caption=false,font=normalsize,labelfont=sf,textfont=sf]{subfig}
%\else
%  \usepackage[caption=false,font=footnotesize]{subfig}
%\fi
% subfig.sty, written by Steven Douglas Cochran, is the modern replacement
% for subfigure.sty, the latter of which is no longer maintained and is
% incompatible with some LaTeX packages including fixltx2e. However,
% subfig.sty requires and automatically loads Axel Sommerfeldt's caption.sty
% which will override IEEEtran.cls' handling of captions and this will result
% in non-IEEE style figure/table captions. To prevent this problem, be sure
% and invoke subfig.sty's "caption=false" package option (available since
% subfig.sty version 1.3, 2005/06/28) as this is will preserve IEEEtran.cls
% handling of captions.
% Note that the Computer Society format requires a larger sans serif font
% than the serif footnote size font used in traditional IEEE formatting
% and thus the need to invoke different subfig.sty package options depending
% on whether compsoc mode has been enabled.
%
% The latest version and documentation of subfig.sty can be obtained at:
% http://www.ctan.org/pkg/subfig




% *** FLOAT PACKAGES ***
%
%\usepackage{fixltx2e}
% fixltx2e, the successor to the earlier fix2col.sty, was written by
% Frank Mittelbach and David Carlisle. This package corrects a few problems
% in the LaTeX2e kernel, the most notable of which is that in current
% LaTeX2e releases, the ordering of single and double column floats is not
% guaranteed to be preserved. Thus, an unpatched LaTeX2e can allow a
% single column figure to be placed prior to an earlier double column
% figure.
% Be aware that LaTeX2e kernels dated 2015 and later have fixltx2e.sty's
% corrections already built into the system in which case a warning will
% be issued if an attempt is made to load fixltx2e.sty as it is no longer
% needed.
% The latest version and documentation can be found at:
% http://www.ctan.org/pkg/fixltx2e


%\usepackage{stfloats}
% stfloats.sty was written by Sigitas Tolusis. This package gives LaTeX2e
% the ability to do double column floats at the bottom of the page as well
% as the top. (e.g., "\begin{figure*}[!b]" is not normally possible in
% LaTeX2e). It also provides a command:
%\fnbelowfloat
% to enable the placement of footnotes below bottom floats (the standard
% LaTeX2e kernel puts them above bottom floats). This is an invasive package
% which rewrites many portions of the LaTeX2e float routines. It may not work
% with other packages that modify the LaTeX2e float routines. The latest
% version and documentation can be obtained at:
% http://www.ctan.org/pkg/stfloats
% Do not use the stfloats baselinefloat ability as the IEEE does not allow
% \baselineskip to stretch. Authors submitting work to the IEEE should note
% that the IEEE rarely uses double column equations and that authors should try
% to avoid such use. Do not be tempted to use the cuted.sty or midfloat.sty
% packages (also by Sigitas Tolusis) as the IEEE does not format its papers in
% such ways.
% Do not attempt to use stfloats with fixltx2e as they are incompatible.
% Instead, use Morten Hogholm'a dblfloatfix which combines the features
% of both fixltx2e and stfloats:
%
% \usepackage{dblfloatfix}
% The latest version can be found at:
% http://www.ctan.org/pkg/dblfloatfix




% *** PDF, URL AND HYPERLINK PACKAGES ***
%
%\usepackage{url}
% url.sty was written by Donald Arseneau. It provides better support for
% handling and breaking URLs. url.sty is already installed on most LaTeX
% systems. The latest version and documentation can be obtained at:
% http://www.ctan.org/pkg/url
% Basically, \url{my_url_here}.




% *** Do not adjust lengths that control margins, column widths, etc. ***
% *** Do not use packages that alter fonts (such as pslatex).         ***
% There should be no need to do such things with IEEEtran.cls V1.6 and later.
% (Unless specifically asked to do so by the journal or conference you plan
% to submit to, of course. )


% correct bad hyphenation here
\hyphenation{op-tical net-works semi-conduc-tor}


\begin{document}
%
% paper title
% Titles are generally capitalized except for words such as a, an, and, as,
% at, but, by, for, in, nor, of, on, or, the, to and up, which are usually
% not capitalized unless they are the first or last word of the title.
% Linebreaks \\ can be used within to get better formatting as desired.
% Do not put math or special symbols in the title.
\title{Trust management for vehicular networks}


% author names and affiliations
% use a multiple column layout for up to three different
% affiliations
\author{\IEEEauthorblockN{Renan Greca}
\IEEEauthorblockA{Departamento de Informática\\
Universidade Federal do Paraná\\
Curitiba, Brazil\\
Email: rdmgreca@inf.ufpr.br}
\and
\IEEEauthorblockN{Luiz Carlos Albini}
\IEEEauthorblockA{Departamento de Informática\\
Universidade Federal do Paraná\\
Curitiba, Brazil\\
Email: albini@inf.ufpr.br}}

% conference papers do not typically use \thanks and this command
% is locked out in conference mode. If really needed, such as for
% the acknowledgment of grants, issue a \IEEEoverridecommandlockouts
% after \documentclass

% for over three affiliations, or if they all won't fit within the width
% of the page, use this alternative format:
% 
%\author{\IEEEauthorblockN{Michael Shell\IEEEauthorrefmark{1},
%Homer Simpson\IEEEauthorrefmark{2},
%James Kirk\IEEEauthorrefmark{3}, 
%Montgomery Scott\IEEEauthorrefmark{3} and
%Eldon Tyrell\IEEEauthorrefmark{4}}
%\IEEEauthorblockA{\IEEEauthorrefmark{1}School of Electrical and Computer Engineering\\
%Georgia Institute of Technology,
%Atlanta, Georgia 30332--0250\\ Email: see http://www.michaelshell.org/contact.html}
%\IEEEauthorblockA{\IEEEauthorrefmark{2}Twentieth Century Fox, Springfield, USA\\
%Email: homer@thesimpsons.com}
%\IEEEauthorblockA{\IEEEauthorrefmark{3}Starfleet Academy, San Francisco, California 96678-2391\\
%Telephone: (800) 555--1212, Fax: (888) 555--1212}
%\IEEEauthorblockA{\IEEEauthorrefmark{4}Tyrell Inc., 123 Replicant Street, Los Angeles, California 90210--4321}}




% use for special paper notices
%\IEEEspecialpapernotice{(Invited Paper)}




% make the title area
\maketitle

% As a general rule, do not put math, special symbols or citations
% in the abstract
\begin{abstract}
By integrating processors and wireless communication units into vehicles, it is possible to create a vehicular ad-hoc network (VANET), in which cars share data amongst themselves in order to cooperate and make roads safer and more efficient.
A decentralized ad-hoc solution, which doesn't rely on previously existing infrastructure, Internet connection or server availability, is preferred so the message delivery latency is as short as possible in the case of life-critical situations.
However, as is the case with most new technologies, VANETs will be a prime target for attacks performed by malicious users, who may benefit from affecting traffic conditions.
In order to avoid such attacks, one important feature for vehicular networks is trust management, which allows nodes to filter incoming messages according to previously established trust values assigned to other nodes.
To generate these trust values, nodes use information acquired from past interactions; nodes which frequently share false or irrelevant data will have lower trust values than the ones which appear to be reliable.
This work proposes a trust management model in the context of daily commutes, utilizing the Working Day Movement Model as a basis for node mobility.
The results prove to be accurate and efficient, thanks to the low complexity of the algorithms constituting the trust model.
\end{abstract}

% no keywords




% For peer review papers, you can put extra information on the cover
% page as needed:
% \ifCLASSOPTIONpeerreview
% \begin{center} \bfseries EDICS Category: 3-BBND \end{center}
% \fi
%
% For peerreview papers, this IEEEtran command inserts a page break and
% creates the second title. It will be ignored for other modes.
\IEEEpeerreviewmaketitle



\section{Introduction}
\label{section:introduction}

Within the next few years, a substantial share of new vehicles will come equipped with networking features \cite{connectedcar2016}.
These features will allow vehicles to quickly share data with other nearby devices and can be useful tools to reduce traffic and the risk of accidents.
Over 1 million people lose their lives to traffic accidents every year \cite{whotraffic}, so solutions to improve road safety are crucial for modern life.
Vehicular ad-hoc networks are a much-studied usage of vehicular networking features.
In these networks, all nodes are related to traffic; they can be vehicles equipped with on-board computers, or stationary units placed near roads.
By quickly sharing data with neighboring vehicles, without the need of an Internet connection, smart vehicles can alert their drivers of important road conditions \cite{barba2012smart}, while autonomous vehicles can synchronize their movements to maximize traffic throughput \cite{amoozadeh2015platoon}.

The accepted for vehicular communication is the IEEE 802.11p or Wireless Access in Vehicular Environments (WAVE) \cite{jiang2008ieee}.
It describes two types of nodes for vehicular networks: on-board units (OBUs) and road-side units (RSUs).
Communications between two OBUs is called vehicle-to-vehicle (V2V) communication, while communication between an OBU and an RSU is called vehicle-to-infrastructure (V2I) communication.
This study focuses only on V2V scenarios, and therefore, any references to VANETs and their nodes refer exclusively to vehicles and their on-board units.

As is expected for new technologies, vehicular communications can become an appealing target for malicious users and attackers.
Some issues that could be exploited in a VANET include: vehicles with faulty sensors \cite{isaac2010security}; vehicles broadcasting false data \cite{golle2004detecting}; a flood of false data to generate a distributed denial of service (DDoS) scenario or to divert traffic \cite{garip2015congestion}; eavesdropping on other vehicles' communications, signal jamming and stalking \cite{isaac2010security}. 

Each of these problems require specific solutions, although there are ways of making the network safer in general.
One way is taking advantage of the concept of trust between members of the network.
By having nodes remember previous interactions with one another, it is possible for them to build trust relationships and avoid those attacks that involve the spread of false data.
Trust solutions for VANETs are generally classified into data-oriented trust, which emphasizes the contents of a message, and entity-based trust, which emphasizes the sender of a message.
The solution described in this article is one of entity-based trust.

The remainder of this paper is organized as follows. \autoref{section:previouswork} shows some examples of existing trust models for vehicular networks and the trust model for static networks that served as basis for this one; \autoref{section:algorithm} explains the algorithms used to develop the trust model; \autoref{section:results} shows the simulations which validate the usefulness of the trust model; and \autoref{section:conclusion} presents closing thoughts on the project.

\section{Previous work}
\label{section:previouswork}

Several models have been proposed to solve the problem of trust in vehicular networks. 
In this section, some of the most relevant ones are described, considering the time in which they were proposed, the advantages they bring and their contributions to later study. 
None of them provide a complete solution, but serve as pieces of a puzzle that is still incomplete. 
Many trust management solutions for VANETs have been proposed over the years, such as \cite{patwardhan2006data}, \cite{gerlach2007trust}, \cite{raya2008data}, \cite{huang2010situation}, \cite{ding2013novel}, \cite{haddadou2013trust}, \cite{liu2016lsot}, \cite{kerrache2016detection}.
There are also some review and/or survey articles on the subject of VANET trust models, such as \cite{zhang2011survey}, \cite{ma2011survey}, \cite{zhang2012trust}, \cite{mejri2014survey}, \cite{soleymani2015trust} \cite{sengar2016survey}, and \cite{dwivedi2016review}.

For the Malicious Node Identification Algorithm (MaNI) proposed in \cite{vernize2015malicious}, the authors present a malicious node identification scheme based on strongly connected components and graph coloring.
The model is proposed for complex networks in general, but is not suited for VANETs because it is designed only for static networks.
Furthermore, the algorithm is executed by a global observer which has information about the complete network.
It is, however, very efficient thanks to the classification of nodes into components and the usage of a fast heuristic.
The usage of components and coloring serve as basis for the trust model proposed here, which is expanded to work on distributed and dynamic networks such as vehicular networks.


\section{Algorithm}
\label{section:algorithm}

The objective of the trust model is to allow nodes to infer whether or not other nodes in the network are malicious.
The algorithm that dictates the trust model runs continuously, with iterations happening in a preset interval.
In every iteration, a node checks its neighbors to see if there were changes to the network and runs a combination of algorithms that help it detect malicious nodes in the known network.
Then, it separates graph $T$ into components using Tarjan's strongly connected components algorithm.
Finally, it uses a graph coloring algorithm as a heuristic to determine which nodes to trust or not.

%In order to do this, a node must maintain a graph $T=(V,O)$ that represents its knowledge of the network, in which $V$ is the set of known nodes and $O$ is the set of opinions nodes have about others.
%Opinions are directed edges, each one with a weight between 0 and 1 that represents the degree of trust from one node towards another.
%A threshold $0 < h < 1$ may be used as a parameter to define the minimum weight for an edge to be considered positive, meaning that the origin node trusts the destination node.


The detailed descriptions of both algorithms are below, followed by the complete process of each iteration of the trust model.

% ---------------------------------------------------------------------------------------------------
\subsection{Tarjan's strongly connected components algorithm}
\label{section:tarjan}
An important aspect of the trust model is the use of Tarjan's strongly connected components algorithm \cite{tarjan1972depth}.
This allows a large graph to be abstracted into a smaller graph, which therefore reduces the input for further algorithms.
Given a directed graph $T = (V,O)$, a strongly connected component is defined as a group of nodes in which, for any pair of nodes $u, v \in V$, there exists a path from $u$ to $v$ and a path from $v$ to $u$.
For the purposes of trust management, this definition is extended to accept only paths of edges with weight 1.
Every node of the input graph $T$ must belong to a component.

In the implementation used, $index$, $lowlink$, $count$ and $stack$ are global variables accessible from every call of the function.
$index$ and $lowlink$ are arrays indexed by node IDs (predefined unique identifiers), $count$ is an integer and $stack$ is a last-in-first-out data structure.

The algorithm works by performing a depth-first search, adding nodes to $stack$ as they are visited.
If two nodes are present on $stack$, then there is a path from the first node to the second one (in the order they were added to $stack$).
Each node has two attributes assigned to it during the execution of the algorithm: $index$ is used to number the nodes in the order they are visited, while $lowlink$ is the lowest indexed node reachable from each node.

In the call that visits a node $u$, the algorithm must loop through each node $v$ trusted by $u$ (that is, $u \rightarrow v$ exists and has value 1).
If node $v$ has not yet been visited, the algorithm is called for  $v$.
The $lowlink$ of $u$ is then calculated as the smallest value between $lowlink[u]$ and $lowlink[v]$, because any node reachable from $v$ is also reachable from $u$.
After the loop, if $lowlink[u]$ is equal to $index[u]$, it means that $u$ is the lowest indexed node reachable from itself and that it is the root of a component.
Therefore, nodes must be popped from the $stack$ until $u$ is found.
Each node popped, including $u$, is a member of a strongly connected component.

The number of components is, at most, $|v|$.
In a worst-case scenario, each node is put into its own component; however, this would not be the case for most useful input graphs.

Algorithm \autoref{algorithm:tarjan} shows the general structure of Tarjan's algorithm  \cite{tarjan1972depth}.
The complexity of the algorithm is $O(|V|+|O|)$ for a graph $T = (V,O)$.

With the results of Tarjan's algorithm, an undirected component graph $C = (V',O')$ is formed.
Each $v' \in V'$ is the abstraction of one component identified by Tarjan's algorithm, while the edges $o' \in O'$ are edges from $T$ between nodes that do not belong in the same component.

\begin{algorithm}
\caption{Tarjan's strongly connected components algorithm}\label{algorithm:tarjan}
\begin{algorithmic}[1]

\Function{Tarjan}{vertex $u$}

\State $index[u] = count$
\State $lowlink[u] = count$
\State $count \gets count + 1$
\State \textbf{push} $u$ \textbf{to} $stack$

\For{$v$ \textbf{in} neighbors of $u$ }
	\If{weight of $u \rightarrow v$ is 0}
		\State \textbf{continue}
	\EndIf	
		
	\If{$index[v] = -1$}
		\Comment{\emph{v has not been visited yet}}
		\State \textbf{Tarjan}($v$)
	\EndIf
	
	\State $lowlink[u] \gets \textbf{min}(lowlink[u],lowlink[v])$
\EndFor

\If{$lowlink[u] = index[u]$}
	\Repeat
		\Comment{\emph{unstack nodes until u is found}}
		\State \textbf{pop} $w$ \textbf{from} $stack$
		\State \textbf{add} $w$ \textbf{to} $component$
	\Until {$w = u$}
\EndIf

\EndFunction
\end{algorithmic}
\end{algorithm}

% ---------------------------------------------------------------------------------------------------
\subsection{Graph coloring with minimum colors}
\label{section:coloring}
The algorithm proposed in \cite{mittal2011graph} is an efficient approach to graph coloring, a classic graph theory problem.
Graph coloring is one of the possible heuristics used to detect malicious nodes after the generation of the component graph using Tarjan's algorithm.
Out of the tested heuristics, it presented the best results, so it has been chosen as the heuristic for the trust model.

The process of graph coloring consists of labeling each node with a color so that no two neighboring nodes share the same color.
This problem has been studied in Computer Science since, at least, 1972 \cite{karp1972reducibility} and has been studied as a classic mathematics problem for even longer \cite{kempe1879geographical}.
It has been proven mathematically that any planar graph can be colored with at most four colors \cite{appel1976every}, but discovering the smallest number of colors necessary to color an arbitrary graph (called the graph's chromatic number) is an NP-hard problem \cite{sanchez1989determining}.

In \cite{mittal2011graph}, the authors propose to color a graph using the minimum possible amount of colors.
Although they do not prove that their algorithm always uses the smallest possible amount of colors, the output is always a correct coloration and the algorithm is nevertheless efficient.
The complexity of the algorithm is $O(|E|)$ for an undirected graph $G = (V,E)$.
As a comparison, the classic DSATUR algorithm for graph coloring has complexity $O(|V|^2)$ \cite{brelaz1979new}.
For the purposes of this study, it is not necessary to prove that the coloring algorithm's output uses the minimum possible number of colors.

Algorithm \autoref{algorithm:coloring} shows the general structure of the graph coloring algorithm  \cite{mittal2011graph} \cite{vernize2013dissertation}.

A limitation of this algorithm is that the edges must be sorted according to node indexes beforehand.
It does not matter which nodes get assigned which indexes, but once they are assigned those numbers, the algorithm must follow the edges in numerical order.
This is demonstrated in \cite{vernize2013dissertation}.

\begin{algorithm}
\caption{Graph coloring with minimum colors}\label{algorithm:coloring}
\begin{algorithmic}[1]

\Function{Coloring}{graph $G$}

\State \textbf{color} all nodes of $G$ \textbf{with} 0
\State $d \rightarrow 0$
\For{$e=(u,v)$ \textbf{in} edges of $G$}
	\If{$u$ and $v$ have the same color}
		\If{$color[v] = d$}
			\State $d \rightarrow d+1$
		\EndIf
		\State $color[v] \rightarrow d$
	\EndIf
\EndFor


\EndFunction
\end{algorithmic}
\end{algorithm}

% ---------------------------------------------------------------------------------------------------
\subsection{Trust management for a vehicular network}
\label{section:trustmanagement}
%\subsection{Malicious Node Identification Algorithm}
%\label{section:mani}

%The basis of the presented trust model is the Malicious Node Identification Algorithm (MaNI) proposed in \cite{vernize2015malicious}, \cite{vernize2013dissertation}.
%The authors present a malicious node identification scheme based on strongly connected components and graph coloring.
%The model is proposed for complex networks in general, but is not suited for VANETs because it is designed only for static networks.
%Furthermore, the algorithm is executed by a global observer which has information about the complete network.

In order to work with dynamic networks, an algorithm must consider snapshots as inputs.
The algorithm runs continuously, running new iterations at a predetermined interval.
Each iteration $i$ is associated with a timestamp, which indicates when the snapshot was taken.
A snapshot $G_i = (V_i, E_i)$ represents the complete topology of the network at the iteration $i$; $V_i$ is the set of nodes participating in the network at that time and $E_i$ is the set of edges connecting pairs of nodes within communication range of each other.
As the network is dynamic, it is expected that each $G_i$ is different from $G_{i-1}$, but also that the changes follow a determined mobility model.

Furthermore, it the algorithm runs in a decentralized fashion, meaning each node in the network runs its own instance of the algorithm.
This is necessary because it is not reasonable to expect a supervisor of the network to have complete knowledge of the nodes and relationships in the network.
Each node starts knowing only about itself and maintains its own abstraction of the network surrounding it.
Every node $u$ has a static, connected and directed trust graph $T^u = (V^u, O^u)$, in which $V^u$ is the set of nodes $u$ is aware of and $O^u$ is the set of trust relationships (opinions) $u$ knows about between members of $V^u$.
Since $T^u$ changes over time, there is a $T^u_i$ for every iteration $i$.

In each iteration, every node $u$ runs the following steps to detect malicious nodes in the network:

\begin{enumerate}
	\item Node $u$ checks who are its neighbors (nodes within communication range).
		  Newly discovered nodes and newly formed edges are added to $T^u_i$.
		  Edges are created with weight 0.5.
	\item Node $u$ tests all of its neighbors to discover which ones can be directly trusted or not.
		  New trust values are computed for the edges using the average between the previous value and either 1 (if the neighbor is trustworthy) or 0 (otherwise).
	\item If a neighbor $v$ is trustworthy, $u$ merges $T^v_{i-1}$ into $T^u_i$, adding nodes and edges that were present on $T^v_{i-1}$ but not on $T^u_{i-1}$.
	\item Tarjan's algorithm is executed to identify the strongly connected components of $T^u_i$, resulting in a component graph $C^u_i$.
	\item The graph coloring algorithm is executed on $C^u_i$ and nodes are identified as benign or malicious according to the same rules as the MaNI algorithm.
\end{enumerate}

During steps 1, 2 and 3, the node is collecting and organizing information.
The storage of an abstraction of the network in the form of a trust graph $T$ is crucial for the remainder of the algorithm.
A prerequisite of step 2 is a test that correctly classifies a neighboring node as benign or malicious.
After these steps, $T^u_i$ is formed, which is then used for the following steps.

Each edge $w \rightarrow v$ in $T^u_i$ is weighted according to the degree of trust $w$ has for $v$ with a value between 0 and 1.
The closer the value is to 1, the more $w$ trusts $v$.
These values are not mutual, so the value of $w \rightarrow v$ can be different from the value of $v \rightarrow w$.

After the collection of data, $T^u_i$ is separated into strongly connected components using Tarjan's algorithm \cite{tarjan1972depth}, which is described in detail on \autoref{section:tarjan}.
For each node in a component, there is a path formed by edges of weight higher than a threshold $0 < t < 1$ to each other node in the same component.
In other words, within a single component, all nodes trust one another directly or indirectly; nodes that do not satisfy this condition are separated into different components.
Each of these components becomes a node of a component graph $C^u_i = (V'^u_i, O'^u_i)$.

The creation of $C^u_i$ simplifies the remaining computation.
Since each vertex $v' \in V'^u_i$ is a component of $T^u_i$ in which all nodes trust each other, for the purposes of identifying malicious nodes, all nodes within each of those components can be treated as the same.
They can either be benign nodes which legitimately trust one another, or malicious nodes colluding with each other.
After the formation of $C^u_i$, one or more heuristics can be used to classify the nodes as benign or malicious.

The authors of \cite{vernize2015malicious} describe the coloring heuristic, which can use a graph coloring algorithm such as DSATUR \cite{brelaz1979new} or the algorithm described in \autoref{section:coloring} \cite{mittal2011graph}.
After running either algorithms with graph $C^u_i$ as input, the color whose nodes in $C^u_i$ represent the most nodes in $T^u_i$ is classified as correct, and all others are classified as malicious.
Once this information in $C^u_i$ is brought back to graph $T^u_i$, it is trivial to label the nodes in $T^u_i$ as either benign or malicious based on their components' classifications.

In the experiments shown in \cite{vernize2013dissertation}, the coloring heuristic shows the most promising results, identifying a high ratio of the malicious nodes in the network.
Other heuristics were experimented with, but were either less effective in detecting malicious nodes, or provided too many false positives.
Therefore, for the purposes of this research, only the coloring heuristic is considered.

%Two types of experiments were made in each network: first, all malicious nodes inverted the edge weights leading to their neighbors; second, malicious nodes randomly inverted or not the weights.
%In the first scenario, the results show excellent precision in most networks, detecting nearly every malicious node.
%Experimenting with the second scenario, the results are less precise, however still promising: with up to 20\% of malicious nodes in the network, the error rate is under 7\%, while with the worst case, 50\% of the network being malicious, the error rate is approximately 15\%.
%
%The authors suggest running the algorithm repeatedly after removing the malicious nodes from the network.
%By doing this, nearly all malicious nodes are detected by it even when randomly changing edge weights.


% ---------------------------------------------------------------------------------------------------

%The MaNI algorithm works well and is efficient for static graphs.
%However, some important modifications had to be made to accommodate dynamic graphs in a decentralized fashion.
%The main changes made are described below.

%First of all, 


%Another important change is on the trust values themselves.
%MaNI uses binary trust (either 0 or 1).
%This has been replaced by a floating-point number between 0 and 1; a predefined threshold defines which values are deemed trustworthy or not.
%This was not the case in the first versions of the trust model; it was added in order to avoid abrupt changes to opinions.
%Since the network is different on each iteration of the algorithm, there were instances in which a change to the topology caused a radical change to the output of the graph coloring algorithm.
%Using a trust value that gradually changes over time, output variations between two iterations became small and gradual.
%
%At the start of an execution, a node knows only about itself.
%From that point, iterations of the algorithm will run in a set interval of $l$ seconds.
%In each iteration (labeled $t$ after the timestamp it occurs on) and for every node $u$, the following steps are executed:





%To mention:
%- Graph building;
%- Change from binary trust to a weighted model;
%- Evolution of trust over time.

%First of all, there is an important change to the graph knowledge used to run the algorithm.
%The complete trust graph is a parameter to the MaNI algorithm, which is executed by a supervisor with global knowledge of the graph.
%
%In this case, given a complete trust graph $T = (V, O)$, each node $u \in V$ has its own vision of the network and has to build a graph, called $T_u$, to represent its surrounding network.
%In each iteration of the algorithm, $u$ receives information from its neighbors about their own graphs.
%For every neighbor $v \in V$, $T_u$ must me merged with $T_v$.
%An edge $u \rightarrow v$ exists if $u$ is aware of $v$.
%Given sufficient iterations, $T_u$ forms a complete or nearly complete vision of the network.

%Alongside $G$, there is the trust graph $T = (V, O)$ which stores trust relationships between pairs of nodes in the form of directed edges.
%Again, each node $u$ has its own $T_u$, which contains the trust information node $u$ knows about.
%When a node $u$ asks for network information from node $v$, it also merges $T_u$ with $T_v$, but only if $u$ trusts $v$.
%If $u$ deems $v$ malicious, any information originating from $v$ cannot be considered valid.


%Furthermore, this implies that the malicious node detection algorithm runs using incomplete graph knowledge.
%This is expected and generally unavoidable in vehicular networks, which have high mobility and can be extremely large.


\section{Results}
\label{section:results}

In order to test the trust model, simulations were made using an implementation of the algorithm in Python.
To generate the input graphs with node mobility, the ONE simulator \cite{keranen2009one} was used in conjunction with the Working Day Movement Model \cite{ekman2008working}, which provides a reasonable imitation of vehicle movement in real life.
Snapshots of the network were taken every 10 simulated seconds, and these snapshots were used as input for the algorithm.

Most of the parameters for the simulator were taken from the article detailing the Working Day Movement Model.
The simulation ran for 86400 seconds (24 hours), with a work day length of 28800 seconds and a standard deviation of departure time of 7200 seconds.
Nodes move between 7 and 10 m/s in an area of approximately 14 km\textsuperscript{2} based on a section of the map of Helsinki.
%The number of nodes and their communication range vary for different simulations.
There is a total of 160 nodes, 150 of which are following the Working Day Movement Model, and the other 10 are moving randomly to simulate vehicles that do not follow daily patterns.
Since this simulation is for vehicles instead of pedestrians, there are no buses in the model and every node is guaranteed to own a vehicle and travel by car.
The parameters regarding offices, meeting spots and shopping were kept intact.

\subsection{Network Density}
The communication range of nodes vary from 10m to 50m, to illustrate the impact of different network densities.
The network density is a value that abstracts the volume and frequency of connections in a vehicular network.
It is calculated using the communication range of the nodes, the amount of nodes, and the total area of the simulation.
For this trust model, higher densities yield better results, since nodes can construct and update their models of the network more quickly (this is demonstrated in \autoref{subsection:simulations}).

The simulations shown here have densities that vary between 0.001 (10m range) and 0.04 (50m range).
As a comparison, the density of São Paulo was calculated as 2.24 with 10m range, a value much higher than what is necessary for the algorithm.

\subsection{Simulations}
\label{subsection:simulations}
To improve readability, all figures in this section follow the same format.
The $X$ axis shows the results of sequential iterations, ranging from 0 to 8639, while the $Y$ axis shows a percentage of all nodes in the network, ranging from 0 to 100.
The blue line represents the percentage of nodes detected out of the complete network.
Magenta is the percentage of malicious nodes in the network (ground truth).
Finally, green represents the nodes correctly identified as malicious (true positives), cyan represents the undetected malicious nodes (false negatives) and red represents the benign nodes incorrectly identified as malicious (false positives).


\autoref{fig:random10}, \autoref{fig:random30} and \autoref{fig:random50} show the results of simulations running with 10\% of nodes acting maliciously, with communications range varying from 10m to 50m.
It is possible to see how the increase in communication range allows the algorithm to converge much sooner, taking over 8000 iterations with 10m range and achieving solid results at just over 1000 iterations with 50m range. 

\autoref{fig:random1} to \autoref{fig:random6} show the variation of results for different amounts of malicious nodes in the network.
By the end of one day, the algorithm is able to detect all malicious nodes when they are up to 30\% of the network.
At 40\%, a small part of malicious nodes are yet to be detected.
At 50\%, as is expected, the results are inconsistent as control of the network is completely divided between benign and malicious nodes; at this point, the network is completely compromised.
The amount of malicious nodes also affects network discovery, since nodes do not trust information from malicious neighbors.

\autoref{fig:random7} shows the execution of the algorithm over the course of 7 days.
Most malicious nodes are identified by the end of the first day; in the following iterations, the algorithm finishes building the network model and sorts out remaining false negative or false positive results.
After iteration 20000, the results are completely consistent.

% ----
%
%\begin{figure}
%\centering
%\includegraphics[width=0.5\lin0.5ewidth]{Network_A_10.0/new_plots/10.png}
%%\includegraphics[width=0.45\linewidth]{Network_A_10.0/new_plots/30.png}
%\caption{Simulation without random nodes and range of 10m.} \label{fig:nonrandom1}
%\end{figure}
%
%\begin{figure}
%\centering
%\includegraphics[width=0.5\linewidth]{Network_A_10.0/new_plots/30.png}
%\caption{Simulation without random nodes and range of 30m.} \label{fig:nonrandom2}
%\end{figure}
%
%\begin{figure}
%\centering
%\includegraphics[width=0.5\linewidth]{Network_A_10.0/new_plots/50.png}
%\caption{Simulation without random nodes and range of 50m.} \label{fig:nonrandom3}
%\end{figure}

% ----

\begin{figure}
\centering
\includegraphics[width=0.5\linewidth]{Network_rA_10.0/new_plots/10.png}
\caption{Simulation with random nodes and range of 10m.} \label{fig:random10}
\end{figure}

\begin{figure}
\centering
\includegraphics[width=0.5\linewidth]{Network_rA_10.0/new_plots/30.png}
\caption{Simulation with random nodes and range of 30m.} \label{fig:random30}
\end{figure}

\begin{figure}
\centering
\includegraphics[width=0.5\linewidth]{Network_rA_10.0/new_plots/50.png}
\caption{Simulation with random nodes and range of 50m.} \label{fig:random50}
\end{figure}

% ----

\begin{figure}
\centering
\includegraphics[width=0.5\linewidth]{Network_rA/10_1}
\caption{Simulation with random nodes, range of 10m and 1\% malicious.} \label{fig:random0}
\end{figure}

\begin{figure}
\centering
\includegraphics[width=0.5\linewidth]{Network_rA/10_5}
\caption{Simulation with random nodes, range of 10m and 5\% malicious.} \label{fig:random1}
\end{figure}

\begin{figure}
\centering
\includegraphics[width=0.5\linewidth]{Network_rA/10_10}
\caption{Simulation with random nodes, range of 10m and 10\% malicious.} \label{fig:random2}
\end{figure}

\begin{figure}
\centering
\includegraphics[width=0.5\linewidth]{Network_rA/10_20}
\caption{Simulation with random nodes, range of 10m and 20\% malicious.} \label{fig:random3}
\end{figure}

\begin{figure}
\centering
\includegraphics[width=0.5\linewidth]{Network_rA/10_30}
\caption{Simulation with random nodes, range of 10m and 30\% malicious.} \label{fig:random4}
\end{figure}

\begin{figure}
\centering
\includegraphics[width=0.5\linewidth]{Network_rA/10_40}
\caption{Simulation with random nodes, range of 10m and 40\% malicious.} \label{fig:random5}
\end{figure}

\begin{figure}
\centering
\includegraphics[width=0.5\linewidth]{Network_rA/10_50}
\caption{Simulation with random nodes, range of 10m and 50\% malicious.} \label{fig:random6}
\end{figure}

% ----

\begin{figure}
\centering
\includegraphics[width=0.5\linewidth]{Network_rA7/10_10}
\caption{Simulation with random nodes, range of 10m and 10\% malicious during 7 days.} \label{fig:random7}
\end{figure}

%\begin{subfigure}{0.20\textwidth}
%\includegraphics[width=\linewidth]{images/coloring/4.png}
%\caption{Edge $(2,3)$ is checked and node $3$ gets the current value of $d$.} \label{fig:coloring5}
%\end{subfigure}
%\hspace*{2cm} % separation between the subfigures
%\begin{subfigure}{0.20\textwidth}
%\includegraphics[width=\linewidth]{images/coloring/5.png}
%\caption{Edge $(2,4)$ is checked and node $4$ gets the current value of $d$.} \label{fig:coloring6}
%\end{subfigure}
%\hspace*{2cm} % separation between the subfigures
%\begin{subfigure}{0.20\textwidth}
%\includegraphics[width=\linewidth]{images/coloring/6.png}
%\caption{Edge $(3,4)$ is checked and node $4$ gets a new color. $d=4$ (yellow).} \label{fig:coloring6}
%\end{subfigure}


% An example of a floating figure using the graphicx package.
% Note that \label must occur AFTER (or within) \caption.
% For figures, \caption should occur after the \includegraphics.
% Note that IEEEtran v1.7 and later has special internal code that
% is designed to preserve the operation of \label within \caption
% even when the captionsoff option is in effect. However, because
% of issues like this, it may be the safest practice to put all your
% \label just after \caption rather than within \caption{}.
%
% Reminder: the "draftcls" or "draftclsnofoot", not "draft", class
% option should be used if it is desired that the figures are to be
% displayed while in draft mode.
%
%\begin{figure}[!t]
%\centering
%\includegraphics[width=2.5in]{myfigure}
% where an .eps filename suffix will be assumed under latex, 
% and a .pdf suffix will be assumed for pdflatex; or what has been declared
% via \DeclareGraphicsExtensions.
%\caption{Simulation results for the network.}
%\label{fig_sim}
%\end{figure}

% Note that the IEEE typically puts floats only at the top, even when this
% results in a large percentage of a column being occupied by floats.


% An example of a double column floating figure using two subfigures.
% (The subfig.sty package must be loaded for this to work.)
% The subfigure \label commands are set within each subfloat command,
% and the \label for the overall figure must come after \caption.
% \hfil is used as a separator to get equal spacing.
% Watch out that the combined width of all the subfigures on a 
% line do not exceed the text width or a line break will occur.
%
%\begin{figure*}[!t]
%\centering
%\subfloat[Case I]{\includegraphics[width=2.5in]{box}%
%\label{fig_first_case}}
%\hfil
%\subfloat[Case II]{\includegraphics[width=2.5in]{box}%
%\label{fig_second_case}}
%\caption{Simulation results for the network.}
%\label{fig_sim}
%\end{figure*}
%
% Note that often IEEE papers with subfigures do not employ subfigure
% captions (using the optional argument to \subfloat[]), but instead will
% reference/describe all of them (a), (b), etc., within the main caption.
% Be aware that for subfig.sty to generate the (a), (b), etc., subfigure
% labels, the optional argument to \subfloat must be present. If a
% subcaption is not desired, just leave its contents blank,
% e.g., \subfloat[].


% An example of a floating table. Note that, for IEEE style tables, the
% \caption command should come BEFORE the table and, given that table
% captions serve much like titles, are usually capitalized except for words
% such as a, an, and, as, at, but, by, for, in, nor, of, on, or, the, to
% and up, which are usually not capitalized unless they are the first or
% last word of the caption. Table text will default to \footnotesize as
% the IEEE normally uses this smaller font for tables.
% The \label must come after \caption as always.
%
%\begin{table}[!t]
%% increase table row spacing, adjust to taste
%\renewcommand{\arraystretch}{1.3}
% if using array.sty, it might be a good idea to tweak the value of
% \extrarowheight as needed to properly center the text within the cells
%\caption{An Example of a Table}
%\label{table_example}
%\centering
%% Some packages, such as MDW tools, offer better commands for making tables
%% than the plain LaTeX2e tabular which is used here.
%\begin{tabular}{|c||c|}
%\hline
%One & Two\\
%\hline
%Three & Four\\
%\hline
%\end{tabular}
%\end{table}


% Note that the IEEE does not put floats in the very first column
% - or typically anywhere on the first page for that matter. Also,
% in-text middle ("here") positioning is typically not used, but it
% is allowed and encouraged for Computer Society conferences (but
% not Computer Society journals). Most IEEE journals/conferences use
% top floats exclusively. 
% Note that, LaTeX2e, unlike IEEE journals/conferences, places
% footnotes above bottom floats. This can be corrected via the
% \fnbelowfloat command of the stfloats package.




\section{Conclusion}
\label{section:conclusion}

While malicious nodes can diminish the usefulness of vehicular networks, the concept of trust as applied in VANETs is a powerful tool for those seeking to reduce the spread of false information as much as possible.
In this paper, a new trust model for vehicular networks was presented, which combines the efficiency of previous algorithms in order to generate fast and accurate results.

As nodes travel across the network and collect more data from neighbor, they are able to form an abstraction of the network which can be used to detect malicious nodes.
By placing nodes into strongly connected component with Tarjan's algorithm, a network containing a large amount of node can be simplified into a much smaller one.
And, using a simple graph coloring algorithm, most malicious nodes stand out by having colors different than the majority of nodes.

The experiments show that vehicles within a network can form a sufficient model of the network in around one day, and by then they are also able to detect nearly every malicious node in the network, with a small amount of false positives.
Since trust amongst nodes in the network is a value ranging between 0 and 1, the trust between two nodes gradually increases or decreases according to the results of each iteration.

% conference papers do not normally have an appendix


% use section* for acknowledgment
%\section*{Acknowledgment}
%
%
%The authors would like to thank...





% trigger a \newpage just before the given reference
% number - used to balance the columns on the last page
% adjust value as needed - may need to be readjusted if
% the document is modified later
%\IEEEtriggeratref{8}
% The "triggered" command can be changed if desired:
%\IEEEtriggercmd{\enlargethispage{-5in}}

% references section

% can use a bibliography generated by BibTeX as a .bbl file
% BibTeX documentation can be easily obtained at:
% http://mirror.ctan.org/biblio/bibtex/contrib/doc/
% The IEEEtran BibTeX style support page is at:
% http://www.michaelshell.org/tex/ieeetran/bibtex/
\bibliographystyle{IEEEtran}
% argument is your BibTeX string definitions and bibliography database(s)
\bibliography{IEEEabrv, article}
%
% <OR> manually copy in the resultant .bbl file
% set second argument of \begin to the number of references
% (used to reserve space for the reference number labels box)
%\begin{thebibliography}{1}
%
%\bibitem{IEEEhowto:kopka}
%H.~Kopka and P.~W. Daly, \emph{A Guide to \LaTeX}, 3rd~ed.\hskip 1em plus
%  0.5em minus 0.4em\relax Harlow, England: Addison-Wesley, 1999.
%
%\end{thebibliography}




% that's all folks
\end{document}


