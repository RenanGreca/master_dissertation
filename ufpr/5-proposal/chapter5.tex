\chapter{Proposal}
\label{chap:proposal}

This work proposes that it is possible to adapt a malicious node identification scheme originally designed for static networks to a dynamic environment, such as a VANET.
In order to make this possible, it is necessary to identify features in VANETs that show that nodes can share a long-term relationship, as is the case for social networks.
The arguments for such features are presented in \autoref{section:socialvanets}, while the movement model that enables the simulation of the features is explained in \autoref{section:workingday}.
Through these long-term relationships, it then becomes feasible for nodes to store trust data and share it with other nodes.

By combining a node's own opinions about familiar nodes and trust information received from its neighbors, it is possible to create a model of the surrounding network.
This model includes a trust graph, showing the trust relationships between pair of nodes, which can then be used as input for a malicious node detection algorithm.

%The previous chapters show why vehicular networks require distinguished trust models compared to social networks and traditional MANETs.
%In this chapter, it is proposed that certain aspects from social networks can be found in vehicular networks and, through those aspects, it is possible to adapt a trust model designed for traditional complex networks to a vehicular environment.
%These social properties allow for long-term trust relationships to be established amongst nodes of a network and, therefore, previously existing trust models can be adapted to the vehicular environment.

In this chapter, the details for the process above are presented.
First, it is shown how vehicles can form long-term relationships and trust one another in a similar way to social networks.
Then, the Working Day Movement Model is presented, along with the reasons why it is best fitted for this work.
Next, a previously existing trust model for complex networks is described, including its advantages and reasons why it could be well-fitted for a vehicular environment, followed by a list of changes that must be made to the model in order for it to handle dynamic networks.
Finally, a schedule for the next steps of the project is shown.

\section{Social Networks and VANETs}
\label{section:socialvanets}
Several of the aforementioned articles state that vehicular networks are, by definition, ephemeral.
That is, the likelihood of two nodes meeting each other twice is too low to be relevant.
In this work, it is argued that such claims are not entirely true.
It stands to reason that, throughout the course of several days, many drivers will take similar routes at similar times of day (e.g. to commute to work) and, therefore, their vehicles will be in similar locations each day.
Additionally, many cities rely on main roads to serve as backbones to their traffic, meaning there is a high density of vehicles on those roads during rush hours.
Since that is true for a notable percentage of a city's fleet, it can also be assumed that those vehicles may frequently encounter each other during their commute.
While two vehicles that share a commute route may not be direct neighbors every day, they will likely be relatively close to each other, meaning few hops separate them in the ad-hoc network.
Furthermore, certain pairs of vehicles are bound to be within communication range of each other nearly every day.
Examples of these include vehicles whose owners are drivers or coworkers.
Such vehicles' trust relationship should become steady over time and, in the case of positive trust, they can use each other's information to learn more about other nodes in the network.

%\textbf{[WORKING DAY MOBILITY MODEL]}
%They do not, however, present data regarding specific pairs of nodes.
%[Should we propose to show this information in our work?]

%Regardless of these properties of commuting vehicles, the probability of two specific vehicles meeting each other more than once, in given time, tends to 1.
%So it is also reasonable to assume that two vehicles, whose drivers reside in the same city, over a long period of time (e.g. one year) may meet each other more than once.
%However, if they only meet rarely, their trust opinions about each other may not be very useful.

Most cities also have one or more types of mass transit systems (buses or trains).
Those vehicles can also be part of a VANET and communicate with private cars.
Buses share the same roads as cars, but instead of having specific destinations, they travel a predefined route during the whole day, usually tied to a tight schedule.
Trains travel on rails, so their contact with cars is less frequent, but it can also happen on railroad crossings; they travel long distances in relatively short amount of time, which helps the dissemination of data in a VANET.
In the same way that cars have a high probability of meeting more than once during their commutes, it is also very likely that they will meet the same buses and/or trains frequently.

In \cite{da2013effective}, \cite{cunha2014vehicular}, and \cite{cunha2014possible}, the authors attempt to find features usually attributed to social networks in vehicular networks.
By using a data set from Zurich, they show that some metrics, such as clustering coefficient and number of encounters, have peaks during the rush hours.

%Another way of applying trust models from social networks to VANETs would be to take into account the relationships between drivers.
%If a neighboring vehicle happens to be driven by a close friend or relative, there is reason to believe that it is not a malicious node in the network.
%That might not always be the case, since there may be malicious hackers controlling the node and its driver is oblivious to it.
%In any case, this approach is not covered in this work.

While certification of identity is an issue in vehicular network, this work will not address it.
There are studies focusing on node identification in VANETs (that is, making sure a node is who it claims to be), in particular through the use of a Public Key Infrastructure (PKI) \cite{wasef2010complementing} 
\cite{kumar2015intelligent}.
One study proposes that such identities must be refreshed with a certain frequency to maintain the privacy of each node \cite{golle2004detecting}.
Although it is important that nodes are not able to lie about identity and are able to maintain their privacy, nodes having valid identities that are persistent over time, so that nodes can identify each other over a long period, is a prerequisite for the model proposed here.

\section{Working Day Movement Model}
\label{section:workingday}

To make use of the above properties, it is important to choose a mobility pattern that properly represents the way vehicles move on a daily basis in the real world.
Therefore, the Working Day Movement Model \cite{ekman2008working} will be used.
The model, developed for used in Delay-Tolerant Network (DTN) simulations, includes many of the features that will be necessary to simulate the daily movement of a vehicular network.

As the name implies, the Working Day model abstracts people's movement from their homes to their offices and back.
Each node has a home and a workplace and they need to travel back and forth between those locations on a daily basis.
Occasionally, nodes can also go to other locations for leisure.
Most VANET trust models use the Random Waypoint mobility model, i.e. each node has an origin point, chooses a random location, gets to that location, then chooses another random location and goes there, and so forth.
While this model is efficient for testing trust protocols, it doesn't truly represent vehicle mobility in the real world.
As mentioned above, many drivers have routes they travel on daily, so the Working Day model is a more accurate representation, although it represents the mobility of humans instead of vehicle and requires some adjustments, which will be listed later in this section.

The model proposed by the authors makes use of several other models for specific tasks.
The main mobility model defines nodes and gives them their destinations.
Within it, five other models are used:
\begin{enumerate}
\item 
The \textbf{home activity submodel} describes what nodes do at night, within their homes.
No movement is modeled.
Nodes can be relatives or neighbors, and therefore share the same home.
\item 
The \textbf{office activity submodel} describes the nodes' routines within their offices.
Nodes can go to other locations within the office (such as meeting rooms) and such movement is modeled.
Nodes with share the same office are coworkers.
\item 
The \textbf{evening activity submodel} is responsible for mobility outside the nodes' standard routine. 
They can meet at certain locations (such as restaurants) and spend a few hours gathered with friends.
\item
The \textbf{transport submodel} shows how nodes move around the city.
It includes another tier of submodels, responsible for modeling three different types of transportation: walking, driving, and riding a bus.
Nodes which own a car always use it, while the others can decide to walk or ride a bus depending on the distance between the origin and destination and the available bus stops.
The walking and driving submodels represent similar types of movement, although at different speeds, while the bus submodel follows cyclical routes and can take or deliver passengers at bus stops.
\item
The \textbf{map} represents the city in which the simulation runs.
Its streets constrain the movement of nodes and all homes and offices must be within the map boundaries.
The map can be divided into districts, which increases what the authors define as \textit{locality}.
In the simulation parameters, the number of nodes which reside and work within the same district can be chosen, which means those nodes will rarely leave the district.
Nodes which reside and work in different districts serve to connect the network with their commutes.
\end{enumerate}

By thinking of these submodels for vehicles instead of people, it can become apparent how the frequency and length of encounters between nodes are similar in both instances.
If two vehicles belong to family members or neighbors, they will likely spend most of the night within communication distance, while coworkers' cars spend the office hours close by.
Cars can also meet each other frequently if their drivers go out with friends.
In the vehicular case, there is the added layer of encounters: cars can communicate frequently with buses and other cars that take the same route daily, although the drivers are likely complete strangers.

In the original article, nodes are devices (such as smartphones) being carried by humans.
Therefore, the Working Day model represents not only people's movements inside their cars, but also within their offices, walking on foot, or riding a bus.
To adapt it to a VANET environment, changes need to be made to their definition of node, since they now represent vehicles instead of people. Some of those changes are as follows:
\begin{enumerate}
\item
The office activity submodel no longer needs to model movement within the office and can be identical to the home submodel.
In both, a node can move a small amount once after reaching the office or home, to simulate parking.
This can be done using the Working Day model's parameters.
\item
The walking submodel needs to be disabled, since all nodes will be either cars or buses.
\item
The bus submodel needs to be changed so that each bus is one node in the network, which follows a predefined route with bus stops.
In the original model, each bus could carry several nodes, but this is no longer necessary.
\end{enumerate}
Other necessary changes might become apparent during the development of the study.

One important topic raised in the Working Day movement model article is the use of two metrics: \textit{inter-contact times} and \textit{contact duration}.
The choice of this movement model for tests is more strongly related to inter-contact times, i.e. how much time it takes for two nodes to meet again.
On the other hand, the contact duration is how long each meeting lasts.
For the reasons explained earlier in this section, relatively short inter-contact times is important for the proposed trust model.
Contact duration time is an important metric to measure how much data can be exchanged during each encounter, although taking it into consideration might add excessive complexity to the model.


\section{Original trust model}

The basis of the proposed trust model is \cite{vernize2015malicious}.
This article presents a malicious node identification scheme based on strongly connected components and graph coloring.
The model is proposed for complex networks in general, but is not suited for VANETs because it is designed only for static networks.
Furthermore, the algorithm is executed by a global observer which has information about the complete network.

The input graph $G = (V,E)$ is a static, connected, and directed graph containing all trust relationships in the network.
Such relationships are binary, so there are no varying degrees of trust: either one node trusts another completely (edge value is $1$), or it distrusts the other completely (edge value is $0$).
The relationships are also directed, meaning that if the value of $A\rightarrow B$ is $1$, $B\rightarrow A$ is not necessarily $1$.

The process for identifying malicious nodes within $G$ is as follows:

First, $G$ is separated into strongly components using Tarjan's algorithm \cite{tarjan1972depth}.
In each of these components, all nodes are connected by edges of value $1$.
In other words, within a single component, all nodes trust one another completely; nodes connected by edges of value $0$ are separated into different components.
Each of these components becomes a node of a component graph $T = (V', E')$.

The creation of the graph $T$ simplifies the remaining computation.
Since each node of $T$ is a vertex $v' \in V'$ and each vertex $v'$ is a component of $G$ in which all nodes trust each other, for the purposes of identifying malicious nodes, all nodes within each of those components can be treated as one.
They can either be benign nodes which legitimately trust one another, or malicious nodes colluding with each other.
After the formation of $T$, one or two heuristics can be used to classify the nodes as benign or malicious.

The \textit{coloring heuristic} uses a graph coloring algorithm, either DSATUR \cite{brelaz1979new} or the algorithm proposed in \cite{mittal2011graph}.
Graph coloring is a classic problem of graph theory, in which the objective is to assign each node in a graph a color so that no two neighboring nodes share the same color.
After running either algorithms with graph $T$, the color with the most nodes in $T$ is classified as correct, and all others are classified as malicious.
Once this information in $T$ is brought back to graph $G$, it is trivial to label the nodes in $G$ as either benign or malicious based on their components' classifications.

The other alternative is the \textit{articulation heuristic}.
It uses a graph articulation algorithm, which does a depth-first search and classifies nodes based on their roles in the spanning tree.
A node $n \in V$ is an articulation if its removal from $G$ makes $G$ a disconnected graph or, in other words, increases the number of connected components of $G$.
Once articulation nodes are found, the heuristic classifies each node as correct if it does not have neighbors that are also articulations.

In the experiments shown in the article, the coloring heuristic shows promising results, identifying a high ratio of the malicious nodes in the network.
The coloring and articulation heuristics can be combined, although the experiments using this combination display a high number of false positives.
Therefore, for the purposes of this paper, only the coloring heuristic will be considered.

\section{Necessary changes to the malicious node identification scheme}
\label{section:changes}
While the algorithm described above works well for static graphs, some important modifications will have to be made to accommodate dynamic graphs.
The main changes planned are described below.

\subsection{Trust graph}
The original work's scope does not include the generation of the trust graph, as it is a parameter to the algorithm.
In dynamic graphs, trust relationships change over time, so how those changes affect our trust model have to be considered.
This includes not only the full figure of the graph, but also each node's views on its neighbors.

\subsection{Strongly connected components}
Strongly connected components, or components in which all nodes have positive trust relationships with each other, are somewhat different in dynamic graphs.
As nodes enter, leave, or change locations in the network, the components will change accordingly.
Therefore, our algorithm will have to work using ``snapshots'' of the graph, meaning each measurement will be valid for a specific timestamp.
In a later step, it might be useful to consider how specific nodes may change over several snapshots.

Since the graph changes with time, for every timestamp $t$, there is a corresponding graph $G_t=(V_t, E_t)$, which includes the available vertices and edges in that moment in time.

\subsection{Graph knowledge}
In the original paper, the malicious nodes are identified by a central entity with access to all nodes' trust values.
Not only does our work plan to decentralize this procedure, having complete knowledge of the graph is unfeasible in large-scale dynamic networks such as VANETs.
Therefore, our algorithm needs to work using only a partial understanding of the network.
For this to be viable, each node will likely have to gather information not only from its views on its neighbors, but also from its neighbor's views and so forth.

The authors of \cite{patwardhan2006data} provide important observations regarding the collection of data from local conditions.
Most crucially, they mention how the quality of information is higher when the origin of such information is close to the destination node.
That is, data collected by the node's own sensors have the highest quality, which diminishes as information starts to come from neighbors, neighbors of neighbors and so forth.
They adopt a broadcast model in which RSUs send data about its surroundings and OBUs capture the information.
The trust model proposed here might use a similar broadcast model, although adapted so all nodes are vehicles (OBUs).

Thus, each node will have its own model of the network, and will generate a trust graph $G_i=(V_i, E_i)$, in which $i$ is the node's identifier.
Combined with the necessity of timestamps explained above, this will result in a graph $G_{it}=(V_{it}, E_{it})$ for each node and each timestamp, which will be the input of the new algorithm.
Accordingly, there are also component graphs $T_{it}=(V'_{it}, E'_{it})$, generated on each instance of the algorithm.

During the early stages of development, it will be considered that vertices cannot enter or leave the network and, therefore, only the layout of the edges changes.
This is sufficient to experiment with the algorithm's performance with the added complexity of a dynamic network.


\subsection{Binary trust}
A model using binary trust is one in which trust values can only be 0 or 1.
If node $A$ positively trusts node $B$, the value of the edge $A\rightarrow B$ is 1.
If it does not trust $B$, or is unsure, the value is 0.
Such a scheme makes sense in a static graph but, in a dynamic one, it is interesting to have a range of possible trust values.

Since nodes will interact with each other occasionally and might learn information from other nodes, a range such as $[0,1]$ is useful to express how one trust value can change over time.
As one node learns more about another, the value can grow or diminish and, once it goes above or below a certain threshold, the neighboring node can be deemed trustworthy or not.
The rate in which trust increases or decreases must be experimented with; it could be useful to have trust decrease faster than it increases, but that might also cause false positives in the case of occasional mistakes.

\subsection{Datasets}
The datasets used for experimentation must be changed accordingly.
Instead of using data gathered from social networks, which are static, dynamic network datasets will need to be used, such as traffic information for cities.

At first, the objective is to test using the Working Day Movement Model, setting parameters that contribute to the validation of the algorithm.
When the simulation begins, a set number of nodes will be malicious and, therefore, other nodes will have low trust in them after interacting.
In time, benign and malicious nodes can change their behavior and, accordingly, their trustworthiness will be updated.

It might still be useful to simulate using real-world movement traces.
An example of such dataset is the Zurich Trace \cite{zurichtrace}, which describes the vehicular traces in the city of Zurich, Switzerland.
The chosen dataset doesn't necessarily need to be created for VANETs, since with vehicular traces it is possible to assume all or a percentage of the traced vehicles are smart and can be nodes of a hypothetical VANET.
The mobility of the nodes is the most important aspect of the dataset.

\section{Tasks and schedule}

Given the proposed trust management scheme, the following list of tasks has been developed:

\begin{enumerate}
	\item \textbf{Changes to the malicious node identification algorithm}. Develop and adapt the algorithm using the changes listed in \autoref{section:changes}.
	\item \textbf{Changes to the movement model}. Develop and adapt the Working Day Movement Model according to \autoref{section:workingday}.
	\item \textbf{Study and selection of a simulator}. Several simulators for vehicular networks exist, and one must be chosen in order to run simulations using the newly developed model. The Working Day Movement Model was tested with the ONE Simulator \cite{keranen2009one}, however it was made for DTNs rather than VANETs. Many VANET simulators, such as TraNS \cite{piorkowski2008trans}, were developed by combining a network simulator (such as ns-2) and a vehicular mobility model (such as SUMO \cite{behrisch2011sumo}). One possible solution is to adapt the Working Day model to an existing network simulator.
	\item \textbf{Simulations and adjustments}. Once the core of the project has been developed and a simulation has been chosen, simulations begin. The models may be tweaked in order to achieve better results, and all data from the process will be collected.
	\item \textbf{Analysis of data and results}. With the data collected from simulations, it will be possible to observe the outcome of the proposed trust model. If tweaks were made to the model during simulations, those changes can be listed, showing which ones provided better results.
	\item \textbf{Writing and submission of article}. Having concluded development, an article will be written explaining the motivations, challenges and results of the study. This article will be submitted to a congress.
	\item \textbf{Writing of dissertation}. Finally, the whole process will be detailed within the dissertation.
\end{enumerate}

An approximate schedule is shown in \autoref{table:schedule}.
The first column displays the task, using the numbers from the list above, while the remaining ones are marked when work on that task is expected for the given month.
The months of April through December are in 2017 and January through March are in 2018.

\begin{table}[h!]
\centering
\begin{tabular}{ |p{0.7cm}||p{0.7cm}|p{0.7cm}|p{0.7cm}|p{0.7cm}|p{0.7cm}|p{0.7cm}|p{0.7cm}|p{0.7cm}|p{0.7cm}|p{0.7cm}|p{0.7cm}|p{0.7cm}| }
 \hline
 \textbf{Task} & Apr & May & Jun & Jul & Aug & Sep & Oct & Nov & Dec & Jan & Feb & Mar\\
 \hline
 \hline
 \textbf{1} & \checkmark & \checkmark & \checkmark & & & & & & & & &\\
 \hline
 \textbf{2} & & & \checkmark & \checkmark & \checkmark & & & & & & &\\
 \hline
 \textbf{3} & \checkmark & \checkmark & \checkmark & \checkmark & & & & & & & &\\
 \hline
 \textbf{4} & & & & & \checkmark & \checkmark & & & & & &\\
 \hline
 \textbf{5} & & & & & \checkmark & \checkmark & & & & & &\\
 \hline
 \textbf{6} &  &  & & & & \checkmark & \checkmark & \checkmark & & & &\\
 \hline
 \textbf{7} &  &  & & & & & & \checkmark & \checkmark & \checkmark & \checkmark & \checkmark \\
 \hline
\end{tabular}
\caption{Expected schedule}
\label{table:schedule}
\end{table}

%Cronograma
%
%1. Adaptação do modelo de nós maliciosos
%2. Adaptação do modelo de mobilidade
%3. Escolha e estudo do simulador
%4. Simulações
%5. Análise de dados e resultados
%6. Escrita e submissão de artigo (procurar congresso)
%7. Escrita da dissertação

%=====================================================
