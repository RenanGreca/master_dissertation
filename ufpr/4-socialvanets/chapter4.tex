\chapter{Special properties of VANETs}
\label{chap:socialvanets}

%=====================================================

As was described above, VANETs are a special type of MANETs, which in turn are technological networks.
However, VANETs feature several unique properties which distinguish them, and the behavior of its members, from other types of networks \cite{yousefi2006vehicular}.
Some of these properties include:

\begin{enumerate}

	\item 	Rapidly changing topology.
			Since the nodes are vehicles, they move frequently and at relatively high speeds.
			Each node's wireless communications also have a certain range, so the other nodes within that range (and, therefore, network neighbors) can change very quickly.
			To build a topology model of its surrounding network, a node must often ping its neighbor to gather information.
			Using velocity and location information, however, it is possible to create reasonable assumptions about future states of the topology.
			
	\item	Node mobility is constrained to a pre-existing grid of roads.
			Within those roads, nodes will usually travel in predictable directions according to local laws and historical data.
			The spaces in the grid, like city blocks, provide a challenge to communication both because of distance and because buildings can cause obstructions to radio transmissions.
			
	\item	VANETs are prone to fragmentation, since a gap in the network topology can make two parts of it unable to communicate with each other.
			Combined with the property above, this fragmentation can appear and disappear frequently, depending on the node density.
			
	\item	Due to the changing topology and possible disconnection, connection with distant nodes is not reliable.
			Therefore, the effective diameter of the network is relatively small for important applications.
			
	\item 	Compared to devices like smartphones, vehicles have no notable power constraints.

	\item 	In certain locations and/or moments, large vehicle density will create a large-scale network, since there will be many nodes concentrated in a relatively small space.
	
	\item 	The topology is susceptible to driver behavior.
			First, this means the topology can occasionally change in unpredictable ways.
			Second, contents of a message sent through the network can alter the driver's behavior and therefore change the topology.
			
\end{enumerate}

\section{Implications for trust management}

Despite the technological similarities between VANETs and MANETs, the differences in node behavior in VANETs creates the necessity of distinguished trust management solutions for those types of network.
Trust or reputation management systems in MANETs are usually developed around properties of MANETs such as: devices with constrained battery power, low-speed or no node mobility, and long contact times during each encounter.
As is described above, such properties are not valid for VANETs.



A few examples of trust management models for MANETs are shown below, and the reasons why they could not be applied to VANETs without significant modifications are explained.

