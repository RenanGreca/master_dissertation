\chapter{Conclusion}
\label{chap:conclusion}

The concept of trust as applied in VANETs is a powerful tool for those seeking to reduce the spread of false information as much as possible.
In this paper, a new trust model for vehicular networks was presented, which combines the efficiency of previous algorithms in order to generate fast and accurate results.
As nodes travel across the network and collect more data from neighbors, they are able to form an abstraction of the network which can be used to detect malicious nodes.
By placing nodes into strongly connected components, a network containing a large amount of node can be simplified into a much smaller one.
Using a simple graph coloring algorithm, most malicious nodes stand out by having different colors than the majority of nodes.
This allows for a low complexity approach to malicious node identification in a dynamic network.

The experiments show that vehicles within a network can form a sufficient model of the network in around one day, and by then they are also able to detect nearly every malicious node in the network, with a very tiny amount of false positives.
As the network changes in shape, nodes acquire more information and are able to make even more accurate classifications of malicious nodes around them. Future work includes the extension of the proposed model to use V2I communications.
