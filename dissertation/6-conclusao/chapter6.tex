\chapter{Conclusion}
\label{chap:conclusion}

In the coming years, vehicular networks or VANETs will an important part of safety and security in transportation, optimizing traffic and reducing the number of accidents.
However, they are an appealing target for entities with malicious intents, creating the need of robust solutions to maintain the integrity of such networks.

The concept of trust as applied in VANETs is a powerful tool for those seeking to reduce the spread of false information among members of a network as much as possible.
In this paper, a new trust model for vehicular networks called TruMan was introduced, which combines the efficiency of previously proposed algorithms in order to generate fast and accurate results.
The solution works in a decentralized fashion and is built for the dynamic environment of vehicular networks, although it could also be adapted to other types of networks.

%As nodes travel across the network and collect more data from neighbors, they are able to form an abstraction of the network which can be used to detect malicious nodes.
%By placing nodes into strongly connected components, a network containing a large amount of nodes can be simplified into a much smaller one.
%Using a simple graph coloring algorithm, most malicious nodes stand out by having different colors than the majority of nodes.
%This allows for a low complexity approach to malicious node identification in a dynamic network.

TruMan was evaluated using mobility data gathered from the ONE simulator using the Working Day Movement Model, which approximates node mobility to that of real-world vehicles.
Several simulations were performed, changing certain parameters to understand how the model performs in different scenarios.
The experiments show that vehicles within a network can form a sufficient abstraction of the network in around one day, and with that information they are able to detect nearly every malicious node in the network, with a very small amount of false positives.
As the network changes in shape, nodes acquire more information and are able to make even more accurate classifications of malicious nodes around them.
With the implementation of information aging, TruMan is also able to detect nodes that start benign and become malicious during the simulation.

A next step for future work would be performing experiments simulating more types of known attacks against vehicular networks, then adapting the model to resist such attacks.
Furthermore, TruMan could be extended to consider more types of information in its algorithm, using, for example, V2I communications with road-side units, vehicular roles, or location and time parameters.
Finally, the concepts of the algorithm could be applied to different types of networks.

