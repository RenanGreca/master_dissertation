\chapter{Simulations}
\label{chap:simulations}

\section{SNAP library}
\label{section:snap}
In order to run the simulations necessary to validate the project, a robust library was required to handle graph data structures.
The Stanford Network Analysis Platform (SNAP) library \cite{snap} was chosen primarily because it is memory efficient.
The simulations require multiple graphs that share the same set of nodes (because each node in the network has its own knowledge of the surrounding network), and the SNAP library uses pointers to nodes and edges, saving memory by not having to duplicate the entire data structure.
It is written in C++, but, for these simulations, the Snap.py Python library was used.

The datasets available from SNAP's website were used for the simulations in \cite{vernize2015malicious} and the static network simulations shown in \autoref{chap:simulations}.

\section{The ONE Simulator}
\label{section:theone}

\section{Working Day Movement Model}
\label{section:workingday}

Most VANET trust models consider the Random Waypoint mobility model, i.e. each node has an origin point, chooses a random location, gets to that location, then chooses another random location and goes there, and so forth.
While this model is efficient for testing trust protocols, it doesn't truly represent vehicle mobility in the real world.

To make use of the properties described in \autoref{section:socialvanets}, it is important to choose a mobility pattern that properly represents the way vehicles move on a daily basis in the real world.
Therefore, the Working Day Movement Model \cite{ekman2008working} (WDM) is useful.
The model, developed for use in Delay-Tolerant Network (DTN) simulations, includes many of the features that are necessary to simulate the daily movement of a vehicular network.

As the name implies, the Working Day Model abstracts people's movement from their homes to their offices and back.
Each node has a home and a workplace and they need to travel back and forth between those locations on a daily basis.
Occasionally, nodes can also go to other locations for leisure.
As mentioned above, many drivers have routes they travel on daily, so the Working Day model is a more accurate representation, although it represents the mobility of humans instead of vehicle and requires some adjustments, which are listed later in this section.

The model proposed by the authors makes use of several other models for specific tasks.
The main mobility model defines nodes and gives them their destinations.
Within it, five other models are used:
\begin{enumerate}
\item 
The \textbf{home activity submodel} describes what nodes do at night, within their homes.
No movement is modeled.
Nodes can be relatives or neighbors, and therefore share the same home.
\item 
The \textbf{office activity submodel} describes the nodes' routines within their offices.
Nodes can go to other locations within the office (such as meeting rooms) and such movement is modeled.
Nodes with share the same office are coworkers.
\item 
The \textbf{evening activity submodel} is responsible for mobility outside the nodes' standard routine. 
They can meet at certain locations (such as restaurants) and spend a few hours gathered with friends.
\item
The \textbf{transport submodel} shows how nodes move around the city.
It includes another tier of submodels, responsible for modeling three different types of transportation: walking, driving, and riding a bus.
Nodes which own a car always use it, while the others can decide to walk or ride a bus depending on the distance between the origin and destination and the available bus stops.
The walking and driving submodels represent similar types of movement, although at different speeds, while the bus submodel follows cyclical routes and can take or deliver passengers at bus stops.
\item
The \textbf{map} represents the city in which the simulation runs.
Its streets constrain the movement of nodes and all homes and offices must be within the map boundaries.
The map can be divided into districts, which increases what the authors define as \textit{locality}.
In the simulation parameters, the number of nodes which reside and work within the same district can be chosen, which means those nodes rarely leave the district.
Nodes which reside and work in different districts serve to connect the network with their commutes.
\end{enumerate}

By thinking of these submodels for vehicles instead of people, it can become apparent how the frequency and length of encounters between nodes are similar in both instances.
If two vehicles belong to family members or neighbors, they likely spend most of the night within communication distance, while coworkers' cars spend the office hours close by.
Cars can also meet each other frequently if their drivers go out with friends.
In the vehicular case, there is the added layer of encounters: cars can communicate frequently with buses and other cars that take the same route daily, although the drivers are likely complete strangers.

In the original article, nodes are devices (such as smartphones) being carried by humans.
Therefore, the Working Day model represents not only people's movements inside their cars, but also within their offices, walking on foot, or riding a bus.
To adapt it to a VANET environment, changes need to be made to their definition of node, since they now represent vehicles instead of people. Some of those changes are as follows:
\begin{enumerate}
\item
The office activity submodel no longer needs to model movement within the office and can be identical to the home submodel.
In both, a node can move a small amount once after reaching the office or home, to simulate parking.
This can be done using the Working Day model's parameters.
\item
The walking submodel needs to be disabled, since all nodes are either cars or buses.
\item
The bus submodel needs to be changed so that each bus is one node in the network, which follows a predefined route with bus stops.
In the original model, each bus could carry several nodes, but this is no longer necessary.
\end{enumerate}
Other necessary changes might become apparent during the development of the study.

One important topic raised in the Working Day movement model article is the use of two metrics: \textit{inter-contact times} and \textit{contact duration}.
The choice of this movement model for tests is more strongly related to inter-contact times, i.e. how much time it takes for two nodes to meet again.
On the other hand, the contact duration is how long each meeting lasts.
For the reasons explained earlier in this section, relatively short inter-contact times is important for the proposed trust model.
Contact duration time is an important metric to measure how much data can be exchanged during each encounter, although taking it into consideration might add excessive complexity to the model.

\section{Simulation parameters and methodology}
\label{section:parameters}

\subsection{Validation}

\section{Restrictions}
\label{section:restrictions}

\section{Results}
\label{section:results}